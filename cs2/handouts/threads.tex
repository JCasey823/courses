\documentclass[12pt]{article}
\usepackage{fullpage}
\usepackage{times}
\usepackage[normalem]{ulem}
\usepackage{fancyhdr,graphicx,amsmath,amssymb, mathtools, scrextend, titlesec, enumitem,}
\usepackage[cache=false]{minted}


\title{Threads}
\author{Alvin Grissom II}
\begin{document}
\maketitle

\textbf{Threads} allow for multiple operations to be run simultaneously.
\begin{minted}[linenos, mathescape, obeytabs=true, frame=lines]{java}
public class ThreadedNums implements Runnable {
     int i, j;
     public static void main(String[] args) {
        ThreadedNums tNums = new ThreadedNums(1,10);
        ThreadedNums tNums2 = new ThreadedNums(500,510);
        Thread thread1 = new Thread(tNums);
        Thread thread2 = new Thread(tNums2);
        thread1.start();
        thread2.start();
    }
    public ThreadedNums(int start, int end) {
        this.i = start;
        this.j = end;
    }
    public void run() {
        while(i < j) {
            System.out.print(i + " ");
            i++;
        }
    }
}
\end{minted}
The program prints the numbers from 1 to 9 and 500 to 509.  First,
notice that we create two \texttt{ThreadedNums} objects:
\texttt{tNums} and \texttt{tNums2}.  The object tNums prints 1-9,
while \texttt{tNums2} prints 500-509.  Then we instantiate two Thread
objects and pass tNums and tNums2 into the Thread's constructor as an
argument.  We can do this only because we implemented
\texttt{Runnable}.  Finally, we start the threads by calling their
respective \texttt{start()} methods.

Output may vary, but the following output is one possibility: 1 2 3 4
5 6 500 501 7 502 8 9 503 504 505 506 507 508 509 The numbers are
clearly out of order, with 7, 8, and 9 appearing between those in the
500 series.  This is possible, because \texttt{tNums1} and
\texttt{tNums2} are executing simultaneously; thus, their respective
for loops are also executing concurrently.  The output, then, is a
conglomeration of the two.  When the \texttt{start()} method of a
thread is called, the code placed in the overridden run() method is
executed, and the code in the calling thread---in this case,
\texttt{main()}---continues as the thread runs in the background.

It is also possible to put a thread to "sleep," but calling its
sleep(int time) method, which causes the thread to cease executing for
a certain number of milliseconds.

Often, in computer programs, we desire that the program work on more
than one task simultaneously.  This is called concurrency, and it is
accomplished using threads.

A thread is part of a program which is working on a task.  Most simple
Java programs have only one thread: the main thread, which executes
the code sequentially.  It is possible, however, and often necessary
to work on more than one task at a time.  This requires multithreaded
programming.  In multithreaded programming, several parts of a program
can run simultaneously, without having to wait for the other sections
to complete their tasks.

 First, consider the following unthreaded Java program: 

\begin{minted}[linenos, mathescape, obeytabs=true, frame=lines]{java}
public class PrintNums {
    public static void main(String[] args) {
        for(int i = 0; i < 10; i++) {
            System.out.print(i + " ");
        }
        for(int j = 500; j < 520; j++) {
            System.out.print(j + " ");
        }
    }
}

\end{minted}
\textbf{Waiting for Threads to Finish}

 Sometimes, we may want to wait for a thread or threads to finish
 before continuing.  For example, we may have delegated parts of a
 task to multiple threads and intend to combine the results when they
 all finish.  Fortunately, Java provides a straightforward way to do
 this, with the \texttt{join()} function in the \texttt{Thread} class.
 
\begin{minted}[linenos, mathescape, obeytabs=true, frame=lines]{java}
Threads t = new Thread(tNums);
t.start();
t.join();
\end{minted}

The call to \texttt{join()} stops the execution of the current program
to wait until \texttt{t} has finished.  By extension, if you call
\texttt{join()} on all of your threads, your program will wait for all
of them to complete before continuing.

There's much more than can be said about threads.  This sheet only
serves as an introduction.
\end{document}
