\documentclass[12pt]{article}
\usepackage{fullpage}
\usepackage{times}
\usepackage[normalem]{ulem}
\usepackage{fancyhdr,graphicx,amsmath,amssymb, mathtools, scrextend, titlesec, enumitem}
\usepackage[cache=false]{minted}
\include{pythonlisting}

\title{Recursion}
\author{Alvin Grissom II}
\begin{document}
\maketitle

\textbf{Recursion} occurs when something is defined in terms of
itself. Consider following program which computes $\sum_{i=1}^{10} x$.
\begin{minted}[linenos, mathescape, obeytabs=true, gobble=2, frame=lines]{java}
    class RecursiveSum {
        public static void main(String[] args) {
            System.out.println(sum(10));
        }
        
        public static int sum(int num) {
            if(num == 0) {
	            return 0;
            }
            return num + sum(num - 1);
        }
    }
\end{minted}
In this example, the function \texttt{sum} calls itself and passes the
result to the new call.  On line 3, \texttt{sum} is passed the value
10.  On line 6, \texttt{prev} is set to 10.  The condition on line 7
evaluates to \texttt{false}; so, we move to line 10.  On line 10, we
add \texttt{prev}, which is 10, to \texttt{sum(prev - 1)}, which
repeats the process, starting with 9.

As a result, we end up with \texttt{sum(10) + sum(9) + sum(8) + sum(7)
  + } $\dots$ \texttt{+ sum(1) + sum(0)}.  Here, \texttt{sum(0)} is
called our \textbf{base case}, because it tells us when to stop.
Without a base case, we would continue to call this function
indefinitely, until we ran out of memory.

Now, let's run through the calculation: We know the
\texttt{sum(0)}$=0$, by definition (lines 7-9).  What about
\texttt{sum(1)}?  Line 10 tells us that
\texttt{sum(1)}$=$\texttt{sum(0)}$+1$. So, what is \texttt{sum(2)}?
\texttt{sum(2)}$=2 + sum(1).$ \texttt{sum(3) = 3 + sum(2)}.  And so
on.  Expanding this, we get:
$$sum(n) = n + sum(n-1)$$ 
$$sum(n-1) = n - 1 + sum(n-2)$$ 
$$sum(n-2) = n - 2 + sum(n-3)$$ 
And so on. \\

If we work this down to \texttt{sum(0)}, we get:

$0+(0 + 1) + (2 + (0 + 1)) + (3 + (2 + (0 + 1))) + (4 +(3 + (2 + (0 + 1)))) +\dots = 55$.

\textbf{Inside of the Computer}

Computer memory is divided into sections that programs can access for
different purposes.  For now, we're going to concerns ourselves
primarily with the \textbf{stack}.  In the Java Virtual Machine (JVM),
we have \textbf{heap} space, which is where objects are stored.
Whenever you create a class, it is stored on the heap.  Think of a
``heap'' of unordered space, from which the computer grabs available
memory.  The \textbf{stack}, however, is a last-in first-out (LIFO)
data structure.  The first item to be placed on the stack is the last
one to be removed.

Functions are placed on the stack, as are primitive variables they
contain.  When you execute a function---for example, \texttt{sum}---it
is placed on the stack in the order in which it was executed. Let's
call it \texttt{sum}$_0$.  If you call \texttt{sum} again, another
call to sum is placed on top of it.  Let's call it \texttt{sum}$_1$.
Since \texttt{sum}$_1$ depends on \texttt{sum}$_0$, \texttt{sum}$_0$
can't complete until \texttt{sum}$_1$ does.  If there's a
\texttt{sum}$_2$, \texttt{sum}$_1$ must wait for it to finish
(return).

This is true of any function called from another function.
%\pagebreak

\begin{minted}[linenos, mathescape, obeytabs=true, gobble=2, frame=lines]{java}
    class StackFunction {
        public void function1() {
            function2();
    	    return;
        }
    
        public void function2() {
        	return;
        }
    }
\end{minted}

In this code, \texttt{function2()} is called from
\texttt{function1()}.  So, \texttt{function1()} is placed on the stack
first.  Then, \texttt{function2()}---which depends on
\texttt{function1()} is called; so, it's placed on the stack next.
When \texttt{function2()} completes, its removed from the stack, and
only \texttt{function1()} remains.  At that point,
\texttt{function1()} can complete, and then it is removed from the
stack.
\end{document}
